\documentclass[12pt]{article}
\usepackage[T1]{fontenc}
\usepackage[utf8]{inputenc}
\title{Rapport de stage}
\author{Anas Alaoui M'Darhri
}
\date{\today}


\begin{document}
\maketitle
\newpage
\section*{Introduction}
\subsection*{Contexte}
CRBM, CNRS, IBC, IGMM, CPBS, LIRMM...\\
Equipe "Bioinformatique Structurale et Modélisation Moléculaire", menée par Dr. Andrey Kajava;\\
Axée sur l'étude de la relation entre séquence, structure et fonctions d'une protéine.\\
\subsection*{Enjeux}
Outils existants mis à la disposition d'autres chercheurs du CRBM.
\subsection*{Objectifs}
Migration de Picasso vers Dali\\
Interfaçage de certains outils\\
"Rafraîchissement" de l'interface\\
 ==> Exemple Tablette (Daniel)\\
++\\
\section{Contexte Métier}
\subsection{Notions de Biologie}
Protéine : Ensemble d'acides aminés.\\
La séquence (la liste) détermine la structure 3D de la protéine (molécule); \\
La structure 3D détermine la fonction;\\
\subsection{Axes de Recherche}
Tandem Repeats : Définition...\\
Amyloïdes : Mécanismes moléculaire, Maladies Liées, Amyloïdes Fonctionnels.\\
\section{Contexte Technologique}
\subsection{Architecture locale}
Picasso32, Dali;
\subsection{Outils de travail}
Ubuntu, Apache2, PostgreSQL+MySQL, Git...
\subsection{Technologies étudiées}
PHP, Perl
\subsection{Machine Learning Algorithms}
\section{Déroulement du Stage}
\subsection{Prise des marques}
Arrivée
\subsection{De Pablo à Salvador}
Migration des outils un par un
\subsection{LA SUITE}
\section{Retour d'expérience}
\subsection{Difficultés rencontrées}
Legacy Code
Pas d'ingénierie Logicielle, Non-Respect des bonnes pratiques...
\subsection{Acquisition de compétences}
Reverse Engineering\\
Administration de DB
Initiation à la recherche
\section*{Conclusion}
\section*{Annexes}
\section*{Bibliographie}

\end{document}
  
